\begin{figure}

\begingroup

\mode
<presentation>

\vspace*{-4.1em}

\mode
<article>

\makeatletter
\let\bfseries\thesis@bfseries@old
\makeatother

\mode
<all>

\centering
\begin{minipage}{0.27\textwidth}

\mode
<presentation>

\vspace*{2.5em}

\mode
<all>

\begin{minted}{xml}
<mrow>
  <mi>x</mi>
  <mo>!!</mo>
  <msup>
    <mi>y</mi>
    <mn>2</mn>
  </msup>
  <mo>=</mo>
  <mn>0</mn>
</mrow>
\end{minted}
\end{minipage}%
\begin{minipage}{0.73\textwidth}
\begin{minted}{xml}
<apply>
  <eq/>
  <apply>
    <minus/>
    <apply>
      <csymbol>double-factorial</csymbol>
      <ci>x</ci>
    </apply>
    <apply>
      <csymbol>superscript</csymbol>
      <ci>y</ci>
      <cn type="integer">2</cn>
    </apply>
  </apply>
  <cn type="integer">0</cn>
</apply>
\end{minted}
\end{minipage}
\endgroup

\mode
<presentation>

\caption
  {Presentation MathML (left) and Content MathML (right)
   representations of $x!! - y^2 = 0$.}

\mode
<article>

\caption
  [Presentation and Content MathML representations]%
  {↑ Presentation MathML (left) and Content MathML (right)
   representations of the math formula $x!! - y^2 = 0$. Notice how the two
   exclamation marks become double factorial operator in Content MathML.
   \cite[Figure 2]{novotny2020three}}

\mode
<all>

\label{fig:pmml-and-cmml}
\end{figure}